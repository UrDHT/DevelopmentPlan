\documentclass[11pt,conference]{IEEEtran}

\usepackage[latin1]{inputenc}
\usepackage{amsmath}
\usepackage{amsfonts}
\usepackage{amssymb}
\usepackage{graphicx}

\title{UrDHT: An Adjective Noun Noun Something}



\author{\IEEEauthorblockN{Andrew Rosen \qquad Brendan Benshoof \qquad Robert W. Harrison \qquad Anu G. Bourgeois}
	\IEEEauthorblockA{Department of Computer Science\\
		Georgia State University\\
		Atlanta, Georgia\\
		rosen@cs.gsu.edu \qquad  bbenshoof@cs.gsu.edu  \qquad rharrison@cs.gsu.edu \qquad anu@cs.gsu.edu }
}

\hyphenation{op-tical net-works semi-conduc-tor Chord-Reduce Map-Reduce Data-Nodes Name-Nodes}


\begin{document}
\maketitle
%TODO:  Actual number of functions
\begin{abstract}
	UrDHT is an abstracted Distributed Hash Table (DHT).
	By completing a few simple functions, a developer can implement the topology of any DHT.
	
	Current distributed systems suffer  from fragmentation , high overhead and inability to scale due to difficulty of adoption.
	UrDHT is P2P system designed to improve the adaptability of P2P distributed serves.
\end{abstract}

\section{Introduction}



\section{What Defines a DHT}

A DHT requires the following functions

\begin{description}
	\item[A \texttt{distance} function] 
	\item[A \texttt{midpoint} function]
	\item[An \texttt{ownership} function]
\end{description}


\section{UrDHT as a DHT}

	\subsection{UrDHT Components}

	\subsection{Hyperbolic Routing}
	
	\subsection{Chord and Ring Based Topology}
	\subsection{Kademlia and Other Tree Based Topologies}

\section{UrDHT as a Gateway}

\section{Experiments}

\section{Future Work and Conclusions}

\end{document}