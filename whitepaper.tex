\documentclass[11pt,conference]{IEEEtran}

\usepackage[latin1]{inputenc}
\usepackage{amsmath}
\usepackage{amsfonts}
\usepackage{amssymb}
\usepackage{graphicx}

\title{UrDHT: A Unified Model for Distributed Hash Tables}



\author{\IEEEauthorblockN{Andrew Rosen \qquad Brendan Benshoof \qquad Robert W. Harrison \qquad Anu G. Bourgeois}
	\IEEEauthorblockA{Department of Computer Science\\
		Georgia State University\\
		Atlanta, Georgia\\
		rosen@cs.gsu.edu \qquad  bbenshoof@cs.gsu.edu  \qquad rharrison@cs.gsu.edu \qquad anu@cs.gsu.edu }
}

\hyphenation{op-tical net-works semi-conduc-tor Chord-Reduce Map-Reduce Data-Nodes Name-Nodes}


\begin{document}
\maketitle
%TODO:  Actual number of functions
\begin{abstract}
	UrDHT is an abstracted Distributed Hash Table (DHT).
	By completing a few simple functions, a developer can implement the topology of any DHT.
	
	Current distributed systems suffer  from fragmentation , high overhead and inability to scale due to difficulty of adoption.
	UrDHT is P2P system designed to improve the adaptability of P2P distributed serves.
\end{abstract}

\section{Introduction}

Distributed Hash Tables have been extensively researched for the past decade.
Despite this, no one has created a cohesive specification for what a DHT is. % or something


UrDHT is our specification and implementation of an abstract DHT.
UrDHT uses a single 


%TODO DO THIS FIRST
\begin{itemize}
	\item We first discuss our motivation for creating UrDHT and \textit{creating it the way we did}  (Section \ref{sec:motivation}).
	\item We
\end{itemize}

\section{Motivation}
\label{sec:motivation}

\section{What Defines a DHT}

A DHT requires the following functions

\begin{description}
	\item[A \texttt{distance} function] \qquad
	This measures distance in the overlay formed by the Distributed Hash Table.
	In most DHTs, the distance in the overlay has no correlation with real-world attributes.
	This is not the case with UrDHT (see Section \ref{sec:hyper}).
	
	\item[A \texttt{midpoint} function]  % Is this part of ownership?
	\item[An \texttt{ownership} definition]
\end{description}


A DHT also needs a strategy to organize and maintain two lists of peers: \textit{short peers} and \textit{long peers}.
Short peers 
These define the topology of the network and guarantee that greedy routing works.

Long peers allow the DHT 
\section{UrDHT}

	\subsection{UrDHT Components (or maybe logic)}
	
	UrDHT is sectioned off into 3 components: database, network, and logic.
	Database handles file storage and network dictates the protocol for how nodes communicate.

	
	\subsection{Hyperbolic Routing}
	\label{sec:hyper}
	
	\subsection{Implementing Chord and Ring Based Topology}
	\subsection{Implementing Kademlia and Other Tree Based Topologies}
	
	\subsection{ZHT}
\section{Experiments}

\section{Future Work and Conclusions}

\end{document}